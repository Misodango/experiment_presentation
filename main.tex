\RequirePackage{plautopatch}
\RequirePackage[l2tabu, orthodox]{nag}

\documentclass[uplatex, dvipdfmx]{jlreq}


% ネットワーク構成図描画用
\usepackage{tikz}
\usetikzlibrary {arrows.meta}
\usetikzlibrary {bending}
\usetikzlibrary {calc}
\usetikzlibrary{angles}

% 省略記号
\def\ellipsis{$\cdots$}

\title{TikZによるニューラルネットワーク作図例}
\author{永松日月}
\date{\today}

\begin{document}
\maketitle

\section{双方向LSTM (Bi-directional LSTM) のネットワーク構成図}
TikZは \LaTeX 内で使える作図用のコマンド群です. これを使って双方向LSTM (Bi-directional LSTM) のネットワーク構成図を作図すると, 図\ref{Fig:Bi-directional LSTM}のようになります.

% ネットワーク構成図
\begin{figure}
\centering

% ノード間の距離
\def\nodeDistance{1.1cm}

% 曲線の始点の角度 (°)
\def\angleStartCurve{360 * 9 / 56}

\begin{tikzpicture}[shorten >= 1pt, ->, node distance=\nodeDistance]
% 変数や省略部のスタイル
\tikzset{var/.style={minimum size=15pt}};

% ノードのスタイル
\tikzset{neuron/.style={var, draw, circle}};

% 入力部を描画
\node[var] (I-n) {$x_n$};
\node[var, left of=I-n] (I-3) {\ellipsis};
\foreach \i/\j in {2/3, 1/2}
{
    \node[var, left of=I-\j] (I-\i) {$x_\i$};
}

% 省略部のノードや変数を描画
\foreach \f/\t in {I/F-LSTM, F-LSTM/B-LSTM, B-LSTM/L-S}
{
    \node[var, above of=\f-3] (\t-3) {\ellipsis};
}
\node[var, above of=L-S-3] (O-3) {\ellipsis};

\foreach \n in {1, 2, n}
{
    % 表示されるノードや変数を描画
    \foreach \o/\f/\t in {neuron/I/F-LSTM, neuron/F-LSTM/B-LSTM, neuron/B-LSTM/L-S}
    {
        \node[\o, above of=\f-\n] (\t-\n) {};
    }
    \node[var, above of=L-S-\n] (O-\n) {$y_\n$};

    % 縦方向の線を描画
    \draw (I-\n) to [out=\angleStartCurve, in=360 - \angleStartCurve] (B-LSTM-\n);
    \draw (F-LSTM-\n) to [out=180 - \angleStartCurve, in=180 + \angleStartCurve] (L-S-\n);
    \foreach \f/\t in {I/F-LSTM, B-LSTM/L-S, L-S/O}
    {
        \path (\f-\n) edge (\t-\n);
    }
}

% 横方向の線を描画
\foreach \i/\j in {1/2, 2/3, 3/n}
{
    \foreach \n/\f/\t in {F-LSTM/\i/\j, B-LSTM/\j/\i}
    {
        \path (\n-\f) edge (\n-\t);
    }
}

% 注釈を描画
\foreach \f/\n in {I/Input, B-LSTM/Backward LSTM, F-LSTM/Foward LSTM, L-S/Linear, O/Output}
{
    \node[node distance=2 * \nodeDistance and \nodeDistance, text width=9em, text centered, right of=\f-n] (A-\f) {\n};
}
\end{tikzpicture}
\caption{双方向LSTM (Bi-directional LSTM)}
\label{Fig:Bi-directional LSTM}
\end{figure}


\begin{figure}
\centering

\begin{tikzpicture}

\fill[black] (0,0) circle (2pt) node[above right] {$Q$};

\foreach \angle in {0,30,...,330}
{
  \draw[->]({cos(\angle)},{sin(\angle)}) --
        ({1.8*cos(\angle)},{1.8*sin(\angle)});

  \draw[->]  ({2*cos(\angle)},{2*sin(\angle)}) --
        ({2.6*cos(\angle)},{2.6*sin(\angle)});

  \draw[->] ({3*cos(\angle)}, {3*sin(\angle)})--
            ({3.4*cos(\angle)}, {3.4*sin(\angle)});
}

\node at (3.5,0) {$\vec{E}$};
\end{tikzpicture}
\end{figure}

\begin{figure}
\centering


\begin{tikzpicture}

\node[align=center] at (-2,3) {誘電率均一 \\ $\varepsilon$};

\draw[->] (-3,0) --(3,0) node[right]{$x$};
\draw[->] (0,-1)--(0,4) node[left]{$y$};
\draw (0,0) node[above right]{O};

\coordinate (Q) at (-2,0);
\coordinate (NQ) at (2,0);
\coordinate (O) at (0,0);
\coordinate (P) at (0.8,2.7);

 % E1とE2の計算
\coordinate (E1) at ($(P)!0.3!180:(Q)$);
\coordinate (E2) at ($(P)!0.4!(NQ)$);

 % E1P = P から E1 へのベクトル
\coordinate (E1P) at ($(E1) - (P)$);

% E_p = E1P + PE2 = (E1 - P) + (E2 - P) + P
\coordinate (Ep) at ($(E1P) + (E2)$);


\fill[blue!20] (NQ) circle(2pt);
\node[below right] at(NQ) {$-Q$};

\fill[blue!20] (Q) circle(2pt);
\node[below left] at (Q) {$+Q$};

\fill[black] (O) circle(1pt);  % 原点を小さめに表示
\fill[black] (P) circle(2pt) node[above left]{$P$};

% 原点から各点電荷への曲線と距離表示
\draw[black, ultra thin] (O) .. controls +(-1,-0.25) and +(1,0.-0.25) .. (Q)
    node[midway, below]{$a$};
\draw[black, ultra thin] (O) .. controls +(1,-0.25) and +(-1,-0.25) .. (NQ)
    node[midway, below]{$a$};

\draw[black] (Q) -- (P) node[midway, left]{$r_1$};
\draw[black] (NQ) --(P) node[midway, right]{$r_2$};

 % ベクトル描画
\draw[-Stealth] (P) -- (E1) node[above right] {$E_1$};
\draw[-Stealth] (P) -- (E2) node[left] {$E_2$};
\draw[-Stealth] (P) -- (Ep) node[right] {$E_p$};

\draw[dotted] (E1) -- (Ep);
\draw[dotted] (E2) -- (Ep);

\draw pic[draw=black, angle radius=4mm] {angle=E2--P--E1};

\node at ($(P) + (0.55,0.2)$) {$\theta$};

\end{tikzpicture}
\end{figure}















\begin{figure}
\centering

\begin{tikzpicture}[scale=1]
\clip (-3,-3) rectangle(3,3);
    % 座標軸の設定
    \coordinate (SX) at (-3, 0);
    \coordinate (GX) at (3, 0);
    \coordinate (SY) at (0, -3);
    \coordinate (GY) at (0, 3);

    % 電荷の位置
    \coordinate (Q1) at (-0.5, 0);
    \coordinate (Q2) at (0.5, 0);


    % 座標軸の描画
    \draw[very thin] (SX) -- (GX);
    \draw[very thin] (SY) -- (GY);

    \draw[<->] (0.5,2.5) -- (0,2.5) node[above, midway] {$a$};
    \draw[<->] (-0.5,2.5) -- (0,2.5) node[above, midway] {$a$};

    \foreach \C in {-3.5,-3,...,-1.5,1.5,2,2.5,3,3.5} {
    \draw[dotted]
        plot[domain=0:360, samples=200]
        ({pow(abs(\C*cos(\x)),2/3)*cos(\x)},
         {pow(abs(\C*cos(\x)),2/3)*sin(\x)});
    }



    \foreach \C in {-3.5,-3,...,-1.5,1.5,2,2.5,3,3.5,6,12} {
    \draw[black!40]
        plot[domain=0:360, samples=200]
        ({pow(abs(\C*cos(\x)),2/3)*sin(\x)},
         {pow(abs(\C*cos(\x)),2/3)*cos(\x)});
}

  % 電荷の描画
    \fill[blue] (Q1) circle(2pt); \node[below left] at (Q1) {$+Q$};
    \fill[blue] (Q2) circle(2pt); \node[below right] at (Q2) {$-Q$};


    \node[right] at (1,2.4) {等電位線};
    \node[right] at(1,0) {電気力線};




\begin{figure}
\centering

\begin{tikzpicture}[scale=1]
\clip (-3,-3) rectangle(3,3);
    % 座標軸の設定
    \coordinate (SX) at (-3, 0);
    \coordinate (GX) at (3, 0);
    \coordinate (SY) at (0, -3);
    \coordinate (GY) at (0, 3);

    % 電荷の位置
    \coordinate (Q1) at (-0.5, 0);
    \coordinate (Q2) at (0.5, 0);


    % 座標軸の描画
    \draw[very thin] (SX) -- (0,0);
    \draw[very thin] (SY) -- (GY);

    \draw[dotted] (-0.5,-3) -- (-0.5,3);
    \draw[<->] (-0.5,2.5) -- (0,2.5) node[above, midway] {$a$};

    \foreach \C in {-3.5,-3,...,-1.5,1.5,2,2.5,3,3.5} {
    \draw[dotted]
        plot[domain=90:270, samples=200]
        ({pow(abs(\C*cos(\x)),2/3)*cos(\x)},
         {pow(abs(\C*cos(\x)),2/3)*sin(\x)});
    }



    \foreach \C in {-3.5,-3,...,-1.5,1.5,2,2.5,3,3.5,6,12} {
    \draw[black!40]
        plot[domain=180:360, samples=200]
        ({pow(abs(\C*cos(\x)),2/3)*sin(\x)},
         {pow(abs(\C*cos(\x)),2/3)*cos(\x)});
}

  % 電荷の描画
    \fill[blue] (Q1) circle(2pt); \node[below left] at (Q1) {$+Q$};

    \node[left] at (-1,2.4) {等電位線};
    \node[left] at(-1,0) {電気力線};

    \foreach \a in {-3,-2.8,...,3}{
    \draw[black!40] (0,\a) -- (0.2,\a+0.2);

    }

\end{tikzpicture}
\end{figure}
\end{document}
